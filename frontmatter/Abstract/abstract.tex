{
% Prevent a new chapter from breaking the page
\let\clearpage\relax

\chapter*{Resumen}\addcontentsline{toc}{chapter}{Resumen - Abstract}

Bajo la validez de ciertas teorías BSM, como los modelos de partículas tipo-axión (Axion-Like Particles), durante una colisión protón-protón en el LHC del CERN, se podría producir una resonancia pseudoescalar de bajas masas $X$, acoplándose principalmente con partículas de la tercera generación del SM. Estas resonancias podrían ser detectadas en ATLAS, siendo producidas en asociación a un par de quarks top y anti-top, y decayendo a un par de leptones tau, de los que se considerará el canal de decaimiento hadrónico {\large(}$t\bar{t}(X \to \tau_{\text{had}} \tau_{\text{had}})${\large)}. En el rango de masas de \num{20} a \SI{60}{\GeV}, el decaimiento $X \to \tau\tau$ presenta dificultades técnicas, al encontrarse los leptones muy próximos entre si, por lo que no pueden resolverse individualmente en el detector. Los más grandes desafíos de esta búsqueda experimental radican en la reconstrucción y caracterización de estos decaimientos \textit{boosteados} de la partícula $X$ para su identificación, y en la estimación de los fondos de eventos del SM que puedan contaminar la señal de Nueva Física. Con este fin, en este trabajo se definen regiones de control y validación para el fondo dominante $t\bar{t}$, estimando los factores de normalización mediante una comparación de las simulaciones MC y datos del año 2017 obtenidos por ATLAS.


\vfill


\chapter*{Abstract}

Under the validity of certain BSM theories, like models involving Axion-Like particles, during a proton-proton collision at the LHC, a low-mass pseudoscalar resonance $X$ can be produced, coupling mainly with third-generation SM particles. These resonances could be detected by ATLAS, produced in association with a pair of top and anti-top quarks, and decaying to a pair of tau leptons, considered to decay hadronically {\large(}$t\bar{t}(X \to \tau_{\text{had}} \tau_{\text{had}})${\large)}. The $X \to \tau\tau$ decay channel presents technical difficulties in the range of masses between \num{20} and \SI{60}{\GeV} since the individual tau leptons can't be resolved in the detector due to a low angular separation. The main challenges in this experimental search are the reconstruction and characterization of the $X$ particle's boosted decays for their proper identification, and the estimation of the SM background events that could contaminate the New Physics signal. With these objectives, in this thesis, we will define control and validation regions for the dominant $t\bar{t}$ background, estimating the normalization factors by comparing the MC simulations and 2017 data recorded by ATLAS.

\vfill
\vfill
}

\clearpage{}

\cleardoublepage{}