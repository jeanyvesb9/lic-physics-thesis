% recto: titlepage 
\thispagestyle{empty}
{
    \calccentering{\unitlength}
    \begin{adjustwidth*}{\unitlength}{-\unitlength}
        \raggedleft{}
        {\Huge\color{Burgundy}%
        Búsqueda de nueva física a través de la producción de resonancias escalares de baja masa en colisiones pp del LHC con el detector ATLAS}\\[\baselineskip]
        {\LARGE%
        Tesis de Licenciatura}\\[0.2\textheight]
        {\huge%
        Jean Yves Beaucamp}\\[\baselineskip]
        {\LARGE%
        28 de Marzo, 2022}
        %
        \vfill
        \vfill
        %
        {\large%
        requirimiento para obtener el grado académico de\\
        Licenciado en Física\\[\baselineskip]
        %
        \vfill
        %
        \begin{tabularx}{0.695\textwidth}{ L{20mm} R{50mm} }
            Directora  & Prof.\ Dra. María Teresa Dova\\
            Codirector & Prof.\ Dr.\; Hernán Wahlberg
        \end{tabularx}
        %\begin{tabularx}{0.787\textwidth}{ L{30mm} R{50mm} }
        %    Directora  & Prof.\ Dra. María Teresa Dova\\
        %    Asesor Académico & Prof.\ Dr.\; Hernán Wahlberg
        %\end{tabularx}
        %
        \vfill
        %
        Departamento de Física,\\
        Facultad de Ciencias Exactas\\[2\baselineskip]
        }
        {\includegraphics[width=5cm]{Assets/Images/UNLP.pdf}}
        \vspace{\baselineskip}
    \end{adjustwidth*}
}

\clearpage{}

% verso: colophon
\thispagestyle{empty}
\hphantom{.}
\vfill

\section*{\normalsize Formato del trabajo}

El formato y diseño del presente documento está inspirado en las publicaciones de Edward Tufte\sidenote{\url{https://www.edwardtufte.com/}}, utilizando como referencia los diseños de las clases \texttt{tufte-latex} y \texttt{classicthesis}, por André Miede.\\

El código fuente se encuentra disponible en \emph{GitHub}\sidenote{\url{https://github.com/jeanyvesb9/lic-physics-thesis}}.

\section*{\normalsize Contacto}

Email: {\small \href{mailto:jean.yves.beaucamp@cern.ch}{jean.yves.beaucamp@cern.ch}}

\clearpage{}

\cleardoublepage{}