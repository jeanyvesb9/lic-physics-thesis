\chapter*{Conclusiones}\addcontentsline{toc}{chapter}{Conclusiones}
\markboth{Conclusiones}{Conclusiones}

En esta tesis se contribuyó al desarrollo de la estrategia general del análisis de búsqueda de una resonancia pseudoescalar $X$, en el rango de masas de \num{20} a \SI{60}{\GeV}, producida en colisiones protón-protón en el LHC. Acoplando con partículas de la tercera generación del SM, la partícula $X$ podría ser observada por ATLAS en el proceso $t\bar{t}(X \to \tau_{\text{had}} \tau_{\text{had}})$. En particular, este canal representa un gran desafío experimental, al encontrarse los leptones \ttau en el estado final muy próximos entre sí, requiriendo el desarrollo de nuevas técnicas para su reconstrucción e identificación conjunta como un único objeto. Los nuevos objetos \textit{DiTau} utilizan técnicas de subestructura de jets para su reconstrucción, requiriendo una implementación especial debido a las características cinemáticas del estado final de la señal en este análisis. La identificación de los objetos DiTau emplea un BDT, cuya puntuación en cada candidato DiTau permite discriminar entre los objetos de señal, producidos por dos \thads, y objetos \textit{fakes}, producidos principalmente por jets de QCD mal identificados. La selección final de eventos, su análisis y la producción de histogramas se realizó con un software basado en ROOT e implementado en C++ y Python, desarrollado como parte del presente trabajo de tesis.

Utilizando una preselección general de eventos, se identificó al proceso $t\bar{t}$ como contribución dominante a los fondos del SM. Se definió una región de control para la normalización global de sus simulaciones de MC mediante una comparación con los datos experimentales, realizando una estimación preliminar de los factores de corrección necesarios para utilizar las predicciones de los fondos en las regiones donde se espera la señal de nueva física. El canal combinado, incluyendo todos los eventos seleccionados en la CR, posee un valor preliminar del factor de transferencia de $\mu_{t\bar{t}}^{\text{Combined}} = 0.885(13)$. El canal muónico alcanza un mejor acuerdo entre las simulaciones del SM y los datos, donde $\mu_{t\bar{t}}^{\mu-\text{channel}} = 0.903(16)$, comparado con $\mu_{t\bar{t}}^{e-\text{channel}} = 0.851(22)$ en el canal electrónico. Por lo tanto, se continuará el análisis de los canales por separado, o alternativamente se incorporará esta discrepancia como una incerteza sistemática adicional en el análisis de un único canal. En todos los casos, se observó una buena correspondencia entre los perfiles de las distribuciones cinemáticas simuladas y los datos en la región de control.

Para validar la normalización a las prediccciones del fondo dominante $t\bar{t}$, se definieron dos regiones de validación con cortes de selección similares a las regiones de control y señal, desplazando la selección de puntuación del BDT de los candidatos DiTau a un intervalo medio entre los valores utilizados en la CR y la SR. Se observaron discrepancias entre el número total de eventos de los datos y en los fondos simulados de $19-20\%$ aproximadamente, con menos eventos de MC. El análisis de las distribuciones de las variables cinemáticas permite inferir que las diferencias en eventos con leptones de bajo $p_T$ resultaron mayores al resto de los eventos, pudiendo atribuirse a no haber incluido aun el fondo Multijet.

Las diferencias observadas exceden las incertezas estadísticas consideradas. Sin embargo, con la adición de una incerteza sistemática de $\sim 20\%$ de los eventos de fondo, estas se encontrarán a menos de una desviación estándar en todas las regiones de validación. Aún así, determinar el orígen sistemático de esta discrepancia resulta prioritario para establecer la validez de la técnica de extrapolación de los factors $\mu_{t\bar{t}}$ a distintas regiones del BDT, en particular a la región de señal de nueva física. Una posible estrategia consiste en la realización de una estimación de fondos \textit{data-driven} en una región similar a la SR, invirtiendo los cortes de selección en las cargas de los objetos DiTau, o algunos de los cortes cinemáticos de los objetos físicos en el evento.

Todos los resultados presentados en esta tesis son preliminares. Se esperan cambios conforme se optimicen las estrategias de identificación y selección de los objetos DiTau, se añadan los fondos Multijet y se introduzcan las incertezas sistemáticas del análisis.

\cleardoublepage{}