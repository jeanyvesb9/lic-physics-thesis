\chapter*{Introducción}\addcontentsline{toc}{chapter}{Introducción}
\markboth{Introducción}{Introducción}

Desde sus comienzos, una curiosidad incesante ha impulsado a los seres humanos en una búsqueda sin fin por comprender el enigmáticamente complejo y diverso mundo que nos rodea. Desde los primeros modelos de cuatro elementos de Empédocles, pasando por el descubrimiento de los elementos químicos, los núcleos atómicos, los protones, electrones y quarks, este voraz viaje por entender la naturaleza fundamental de la materia y sus interacciones dio sus frutos durante el siglo XX, con el nacimiento del Modelo Estándar de la Física de Partículas.

Pese a su modesto nombre, la capacidad predictiva del Modelo Estándar (SM) fue capaz de anticipar la existencia de tres partículas elementales: el quark top, el neutrino tauónico y el bosón de Higgs. Esta última fue recién observada experimentalmente en el año 2012, luego de un titánico esfuerzo de más de tres décadas en el diseño, construcción y operación del acelerador de partículas más poderoso de la historia: el Gran Colisionador de Hadrones.

El Modelo Estándar describe y predice el comportamiento de la materia conocida al interactuar con tres de las cuatro fuerzas elementales (interacción fuerte, interacción débil y la interacción electromagnética), con un nivel de exactitud sin precedentes, obteniendo predicciones que acuerdan con mediciones experimentales en menos de una parte por billón, a escalas de energía muy variadas.

Sin embargo, con el paso de los años, el trabajo ininterrumpido de miles de físicos teóricos y experimentales ha comenzado a vislumbrar señales de una Física más allá del Modelo Estándar. Evidencias de esto incluyen la asimetría materia-antimateria en el universo, la detección de neutrinos masivos, la existencia de la hipotética pero necesaria materia oscura, entre otros.

Existen muchas propuestas para extender el SM y dar explicación a estos fenómenos, algunas de las cuales requieren añadir una nueva partícula pseudoescalar, impar ante transformaciones $CP$ de conjugación carga-paridad. En particular, el espacio de fases en regiones de bajas masas, inferiores a\sidenote{Utilizamos unidades naturales ${c = \hbar = 1}$, por lo que la masa y el momento poseen unidades de energía.} \SI{100}{\GeV}, se encuentra todavía mayormente inexplorado. En este contexto, se propone la búsqueda de un bosón pseudoescalar $X$ de bajas masas, entre \SI{20}{\GeV} y \SI{60}{\GeV}, producido en asociación con un par de quarks top, y con decaimiento a un par de leptones taus. Debido a la diferencia de masas entre el par $t\bar{t}$ y el pseudoescalar $X$, la detección del estado final de la interacción representa un gran desafío experimental, al encontrarse los dos taus muy próximos entre sí. Esto impide una correcta reconstrucción e identificación de los objetos físicos, requiriendo nuevas técnicas en las que se trate al par de taus como un único objeto \textit{DiTau} con subestructura.

Para poder llevar adelante una búsqueda de Nueva Física, es necesaria una precisa y exacta estimación de los fondos contaminantes del Modelo Estándar, que se encuentren presentes en la misma región del espacio de fases en la que se espera observar la señal de los modelos BSM (región de Señal). Las simulaciones de Monte Carlo de los procesos del SM no poseen la precisión necesaria, requiriendo el desarrollo de nuevas estrategias utilizando los datos experimentales. Para tal fin, se buscan regiones de Control (en el espacio de fase) dominadas por el fondo que se quiere determinar, con las que se obtienen factores de corrección para las simulaciones, llamados \textit{transfer factors}, los que son verificados en regiones de Validación. Es precisamente el diseño de estas regiones y el estudio de los fondos del SM presentes en cada una, lo que comprendió casi la totalidad del desarrollo del trabajo de diploma.

El desarrollo de parte de las tareas se ha realizado en el contexto del Programa de Verano del CERN, entre los meses de Julio y Octubre de 2021. Los resultados parciales obtenidos durante este período se encuentran en \cite{Beaucamp:2790493}. El presente análisis de búsqueda es un esfuerzo conjunto en el que participan miembros de los groupos de Física de Altas Energías de la Universidad Nacional de La Plata, Tel Aviv University y McGill University.

El trabajo se encuentra estructurado de la siguiente forma. En el \cref{chap:ch1}, describiremos brevemente el Modelo Estándar, las partículas e interacciones que lo constituyen. Daremos también una abreviada introducción a los modelos de partículas pseudoescalares tipo-axión (ALPs), propuestos como solución a muchos de los problemas e inconsistencias actualmente conocidas del SM. En los \cref{chap:ch2,chap:ch3}, describiremos el experimento ATLAS, sus subsistemas y principios de funcionamiento, así como los métodos de reconstrucción e identificación de los objetos físicos medidos por el detector. Se dará especial énfasis a los objetos \textit{DiTau}. El \cref{chap:ch4} desarrolla la estrategia general del análisis. Identificaremos los posibles procesos del SM que contribuyan a los fondos contaminantes de la señal y enunciaremos criterios generales de preselección de los eventos incluidos en el análisis, dando una primera definición de la región de Señal en la que se realizará la búsqueda. El diseño de la región de Control del fondo dominante se trata en el \cref{chap:ch5}, comparando las simulaciones de Monte Carlo con los datos experimentales y estimando el factor de corrección. Finalmente, las correcciones de los fondos simulados serán validadas con los datos en regiones de Validación en el \cref{chap:ch6}. 

\cleardoublepage{}